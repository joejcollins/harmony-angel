
    % !TeX root = FoodFile.tex
    % Content Begins
    
		\begin{menu}{The Very First Menu}
    
    \subsection*{Shopping Lists}
    %
      \begin{shoppinglist}{Vegetables}
      750 g cabbage \\ 
      400 g carrots \\ 
      400 g celery \\ 
      320 g onions \\ 
      700 g onions \\ 
      2  red chillis \\ 
      %
      \end{shoppinglist}%
      \hfil%
      %
    %
      \begin{shoppinglist}{Groceries}
      400 g baked beans \\ 
      400 g canned tomatoes \\ 
      200 g lasagna \\ 
      3 tbsp peanuts \\ 
      4  pitta bread \\ 
      75 g suet \\ 
      75 g tomato puree \\ 
      %
      \end{shoppinglist}%
      \par %
      %
    %
      \begin{shoppinglist}{Meat}
      300 g chicken \\ 
      400 g meat \\ 
      %
      \end{shoppinglist}%
      %
      %
    %
      \begin{shoppinglist}{Dairy}
      150 g cheese \\ 
      2  eggs \\ 
      4  eggs \\ 
      100 ml milk \\ 
      300 ml soured cream \\ 
      %
      \end{shoppinglist}%
      \par %
      \clearpage %
    %
      \begin{shoppinglist}{Check}
      1 tbsp black bean sauce \\ 
      0.5 tsp chilli powder \\ 
      0.5 tsp mixed herbs \\ 
      2 tbsp oil \\ 
      2 tbsp oil \\ 
      2 tbsp oil \\ 
      2 tbsp oil \\ 
      75 g self raising flour \\ 
      1  stock cube \\ 
      %
      \end{shoppinglist}%
      %
      %
    %
      \begin{shoppinglist}{Extra Vegetables}
      750 g cabbage \\ 
      400 g carrots \\ 
      400 g celery \\ 
      320 g onions \\ 
      700 g onions \\ 
      2  red chillis \\ 
      %
      \end{shoppinglist}%
      \par %
      %
    %
    \othershoppinglist{Other Shopping}\hfil%
    \othershoppinglist{Extra Other Shopping}%
    \clearpage
  
    \begin{recipe}{2}{Meat stew and dumplings}%
    
		\begin{ingredients}
		320 g onions (chopped) \\
	400 g meat (sliced) \\
	400 g carrots (chopped) \\
	400 g celery (chopped) \\
	2 tbsp oil  \\
	75 g tomato puree  \\
	0.5 tsp chilli powder  \\
	1  stock cube (crumbled) \\
	75 g self raising flour  \\
	75 g suet  \\
	0.5 tsp mixed herbs  \\
	
		\end{ingredients}
	
    \begin{instructions}
    \item \item 
				In a  fry
				320 g chopped onions,
				400 g sliced meat,
				400 g chopped carrots
				and
				400 g chopped celery
				in
				2 tbsp  oil
				until meat done.
			\item 
				Stir in
				75 g  tomato puree,
				0.5 tsp  chilli powder,
				1  crumbled stock cube,
				500 ml  cold water
				and simmer for 50 minutes.
			\item 
				In a  mix
				75 g  self raising flour,
				75 g  suet,
				0.5 tsp  mixed herbs
				and
				100 ml  cold water
				and knead into 8 balls.
				Put balls in stew, cover and
				simmer for 15 minutes.
			
    \end{instructions}
    \end{recipe}%
  
    \begin{recipe}{2}{Onion and tomato lasagna}%
    
		\begin{ingredients}
		700 g onions (chopped) \\
	2 tbsp oil  \\
	400 g canned tomatoes (chopped) \\
	200 g lasagna  \\
	150 g cheese (grated) \\
	300 ml soured cream  \\
	100 ml milk  \\
	
		\end{ingredients}
	
    \begin{instructions}
    \item 
				In a  fry
				700 g chopped onions
				in
				2 tbsp  oil
				until soft.
			\item 
				Stir in 400 g chopped canned tomatoes
				and warm through.
			\item 
				In a  layer
				200 g  lasagna,
				the tomato mix and
				150 g grated cheese
				3 times.
			\item 
				In a  mix
				300 ml  soured cream
				with
				100 ml  milk
				and pour on top of the lasagna.
			\item 
				Bake  50 minutes.
			
    \end{instructions}
    \end{recipe}%
  
    \begin{recipe}{2}{Chicken with cabbage and peanuts}%
    
		\begin{ingredients}
		3 tbsp peanuts  \\
	2 tbsp oil  \\
	2  red chillis (chopped) \\
	300 g chicken (sliced) \\
	750 g cabbage (sliced) \\
	1 tbsp black bean sauce  \\
	
		\end{ingredients}
	
    \begin{instructions}
    \item \item 
				In a 	fry
				3 tbsp  peanuts
				in
				2 tbsp  oil
				then remove.
			\item 
				Fry 2  chopped red chillis
				and
				300 g sliced chicken
				in the oil until meat is done.
			\item 
				Stir in
				750 g sliced cabbage
				and heat through.
			\item 
				Stir in 
				1 tbsp  black bean sauce
				and fry for 2 minutes.
			\item 
				Sprinkle with the peanuts.
			
    \end{instructions}
    \end{recipe}%
  
    \begin{recipe}{1}{Beans and fried eggs in pitta}%
    
		\begin{ingredients}
		400 g baked beans  \\
	2  eggs  \\
	2 tbsp oil  \\
	4  pitta bread  \\
	
		\end{ingredients}
	
    \begin{instructions}
    \item 
				In a  warm
				400 g  baked beans.
			\item 
				In a  fry
				2   eggs
				in 
				2 tbsp  oil\item 
				Grill
				4   pitta bread
    \end{instructions}
    \end{recipe}%
  
    \begin{recipe}{2}{Egg and vegetable curry}%
    
		\begin{ingredients}
		4  eggs  \\
	
		\end{ingredients}
	
    \begin{instructions}
    \item 
				In  boil
				4   eggs\item 
    \end{instructions}
    \end{recipe}%
  
    \clearpage
    \end{menu}
	
		\begin{menu}{B}
    
    \subsection*{Shopping Lists}
    %
      \begin{shoppinglist}{Vegetables}
      %
      \end{shoppinglist}%
      %
      %
    %
      \begin{shoppinglist}{Groceries}
      %
      \end{shoppinglist}%
      \par %
      %
    %
      \begin{shoppinglist}{Meat}
      %
      \end{shoppinglist}%
      %
      %
    %
      \begin{shoppinglist}{Dairy}
      4  eggs \\ 
      4  eggs \\ 
      4  eggs \\ 
      %
      \end{shoppinglist}%
      \par %
      \clearpage %
    %
      \begin{shoppinglist}{Check}
      %
      \end{shoppinglist}%
      %
      %
    %
      \begin{shoppinglist}{Extra Vegetables}
      %
      \end{shoppinglist}%
      \par %
      %
    %
    \othershoppinglist{Other Shopping}%
    \othershoppinglist{Extra Other Shopping}%
    \clearpage
  
    \begin{recipe}{2}{Egg and vegetable curry}%
    
		\begin{ingredients}
		4  eggs  \\
	
		\end{ingredients}
	
    \begin{instructions}
    \item 
				In  boil
				4   eggs
    \end{instructions}
    \end{recipe}%
  
    \begin{recipe}{2}{Egg and vegetable curry}%
    
		\begin{ingredients}
		4  eggs  \\
	
		\end{ingredients}
	
    \begin{instructions}
    \item 
				In  boil
				4   eggs
    \end{instructions}
    \end{recipe}%
  
    \begin{recipe}{2}{Egg and vegetable curry}%
    
		\begin{ingredients}
		4  eggs  \\
	
		\end{ingredients}
	
    \begin{instructions}
    \item 
				In  boil
				4   eggs
    \end{instructions}
    \end{recipe}%
  
    \clearpage
    \end{menu}
	
    % Content Ends
	